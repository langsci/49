
\addchap{List of relevant papers by Luc Steels}
\label{c:references}

Relevant papers by Luc Steels and co-authors published about the Talking Heads Experiment and subsequent theoretical 
and experimental work. \\

1. \citet{Steels:1995} The `artificial life' route to `artificial intelligence'. Building Situated Embodied Agents. Lawrence Erlbaum Ass, New Haven, 1994.

2. \citet{Steels:95a}. The Biology and Technology of Intelligent Autonomous Agents. NATO ASI Series: Series F, Computer and Systems Sciences, vol. 144, Berlin: Springer-Verlag.

3. \citet{Steels:95b} A self-organizing spatial vocabulary Artificial Life Journal, 2(3), 319-332. 

4. \citet{Steels:96a} Self-Organizing Vocabularies. In Langton, C. and Shimohara, T. (ed.) Proceeding of 
Alife V. The MIT Press. Cambridge, MA. pp. 179-184. 

5. \citet{Steels:96b} Emergent Adaptive Lexicons. In Maes, P. et.al. (ed) From Animals to Animats 4: Simulation of Adaptive Behavior, The MIT Press, Cambridge Ma. pp. 562-567. 

6. \citet{Steels:96c} Perceptually Grounded Meaning Creation. In Tokoro, M.,(ed.) Proceedings of Multi-Agent Systems, ICMAS-2. AAAI Press, Menlo Park. pp. 338-344. 

7. \citet{Steels:97a} The Spontaneous Self-organization of an Adaptive Language. In Muggleton, S., editor, Machine Intelligence 15, Oxford University Press. Oxford. pp. 205-224. 

8. \citet{Steels:97b} The Synthetic Modeling of Language Origins. Evolution of Communication Journal, 1(1):1-34. 

9. \citet{Steels:1998} The Origins of Syntax in Visually Grounded Robotic Agents. In Pollack, M. (ed.) Proceedings of the 15th International Joint Conference on Artificial Intelligence, Morgan Kaufmann Publishers, San Francisco, CA. pp. 1632-1641. 

10. \citet{Steels:97d} Language Learning and Language Contact. In: Daelemans, W., A. Van den Bosh, A. and A. Weijters (eds.)
Workshop Notes of the ECML/MLnet Familiarization Workshop on Empirical Learning of Natural Language 
Processing Tasks, ECML/MLnet. Prague. pp. 11-24. 

11. \citet{Steels:97e} Constructing and Sharing Perceptual Distinctions. In: van Someren, M. and Widmer, G. (eds.) Proceedings of the European Conference on Machine Learning. Springer-Verlag, Berlin. pp 4-13. 

12. \citet{Steels:97f} Synthesising the Origins of Language and Meaning Using Co-evolution, Self-organisation and Level formation. In Hurford, J. and Knight, C. and Studdert-Kennedy, M. (eds.) Approaches to the Evolution of Language: Social and Cognitive bases, Edinburgh University Press. Edinburgh, 1997. pp. 384-404. 

\enlargethispage{1em}
13. \citet{Steels:97g} Ancrage de Jeux de Langage Adaptatifs dans des Agents Robotiques. In A. Cherif. J. and Signorini (ed.) Intelligence Artificielle et Complexite, Paris, 1997. pp. 27-39. 

14. \citet{Steels:97h} Grounding Adaptive Language Games in Robotic Agents. In Harvey, I. and P. Husbands, P. (eds.) Proceedings of the 4th European Conference on Artificial Life, The MIT Press, Cambridge, MA. pp. 474-482.

15. \citet{Steels:98a} Spontaneous Lexicon Change. Proceedings of COLING-ACL 1998, Montreal. pp. 1243-1249. 

16. \citet{Steels:98b} Structural Coupling of Cognitive Memories Through Adaptive Language Games. In Pfeifer, R. and Blumberg, B. and Meyer, J-A and Wilson, S (eds.) From Animals to Animats 5: Proceedings of SAB 98. The MIT Press, Cambridge, CA. pp. 263-269. 

17. \citet{Steels:1998} The origins of syntax in visually grounded robotic agents. Artificial Intelligence, 103:1-24. 

18. \citet{Steels:98d} Stochasticity as a Source of Innovation in Language Games. In Adami, C. and Belew, R. and Kitano, H. and Taylor, C. (eds.) Proceedings of Artificial Life VI, Cambridge, MA, June 1998. The MIT Press. pp. 368-376. 

19. \citet{Steels:98e} The Origins of Ontologies and Communication Conventions in Multi-Agent Systems. Journal of Agents and Multi-Agent Systems, 1(2), 1998. pp. 169-194

20. \citet{Kaplan:98f} An architecture for evolving robust shared communication systems in noisy environments. Proceedings of Sony Research Forum, Tokyo, 1998.

21. \citet{McIntyre:1999} Net-mobile embodied agents. Proceedings of Sony Research Forum, Tokyo, 1999.

22. \citet{Steels:99b} Cognitive Teleportation and Situated Embodiment. In: Floreano, D. and J-D, Nicoud and F. 
Mondada (eds) (1999) Advances In Artificial Life. LNCS 1674/1999, Springer 1999, p. 7. 

23. \citet{Steels:99c} Spatially Distributed Naming Games. Advances in Complex systems, 1(4), 301-323. 

24. \citet{Steels:99d} The Talking Heads Experiment. Volume 1. Words and Meanings. Pre-edition Laboratorium, Antwerpen, 1999.

25. \citet{Steels:99e} The puzzle of language evolution. Kognitionswissenschaft, 8(4), 143-150. 

26. \citet{Steels:99f} Situated grounded word semantics. In Dean, T. (ed.) Proceedings of IJCAI'99, 
Morgan Kaufmann Publishers. San Francisco, CA. pp. 862-867. 

27. \citet{Steels:99g} Amorcage d'une s\'emantique lexicale dans une population d'agents autonomes, ancr\'es et 
situ\'es. In: Amsili, P. (ed.) Traitement automatique du langage naturel, Cargese, Corse. pp. 393-398. 

28. \citet{Steels:1999} Collective learning and semiotic dynamics. In Floreano, D. and 
Nicoud, J-D and Mondada, F., editor, Advances in Artificial Life (ECAL 99) LNCS 1674, Springer Verlag, Berlin. 
pp. 679-688. 

\enlargethispage{1\baselineskip}
29. \citet{Kaplan:00a} Comment les robots construisent leur monde : Experiences sur la convergence des cat\'egories sensorielles. In: Dessalles, J-L. (ed.), Journ\'e ARC Evolution et Cognition, ENST Paris. pp. 13-18. 

30. \citet{Steels:00b} Language as a Complex Adaptive System. In: Schoenauer, M., (ed.) Proceedings of PPSN VI, LNCS. Springer-Verlag, Berlin. pp. 17-26. 

\newpage
31. \citet{Steels:00c} The Emergence of Grammar in Communicating Autonomous Robotic Agents. In Horn, W. (ed.) Proceedings of ECAI 2000, 
IOS Publishing, Amsterdam, pp 764-769. 

32. \citet{Steels:00d} A brain for language. Proceedings of the 3rd Sony SONY CSL Paris Symposium: The ecological brain, Paris, 2000.

33. \citet{Steels:00e} Mirror Neurons and the Action Theory of Language Origins. Proceedings of 'Architectures of the Mind, Architectures of the Brain', Vatican Rome. 

34.\citet{Steels:2000kaplan} Origine et \'evolution du langage : exp\'eriences robotiques. Revue du Palais de la D\'ecouverte, 278:63-67 May 2000.

35. \citet{Steels:00g} The Cultural Evolution of Syntactic Constraints in Phonology. In: Bedau, M., et al.
(eds) Proceedings of the VIIth Artificial life conference (Alife 7), The MIT Press, Cambridge Ma. pp. 382-391. 

36. \citet{Steels:01a} Social learning and verbal communication with humanoid robots. In: Proceedings of the IEEE-RAS International Conference on Humanoid Robots, IEEE Press, Piscataway NJ. pp. 335-342

37. \citet{Steels:01b} The role of language in learning grounded representations. In: Cohen, P. and T. Oates (eds.) Proceedings of the AAAI Spring Symposium on Grounding. AAAI Press, Anaheim, CA. pp. 80-85. 

38. \citet{Steels:01c} Social and cultural learning in the evolution of human communication. In Kimbrough Oller, D. and Griebel, U. and Plunkett, K., editor, Evolution of communication systems : A comparative approach, the MIT Press. Cambridge, MA,2001.

39. \citet{Steels:01d} Linguistique evolutionnaire et Vie Artificielle. Vie Artificielle, 1(1). Hermes Pub. Paris. 

40. \citet{Steels:01e} Social learning and language acquisition. In: McFarland, D. and  O. Holland (eds.) Social Robots. 
Oxford University Press. Oxford, UK, 2001.nn

41. \citet{Steels:01f} Language games for autonomous robots. IEEE Intelligent systems, pp. 17-22. 

42. \citet{Steels:01g} The methodology of the artificial. Behavioral and brain sciences, 24(6): 1077-1078. 

43. \citet{Steels:01h} AIBO's first words: The social learning of language and meaning. 
Evolution of Communication, 4(1):3-32. 

44. \citet{Steels:02b} Grounding symbols through evolutionary language games. In: Cangelosi, A. and D. Parisi (eds.) 
Simulating the evolution of language, Springer-Verlag, Berlin. pp. 211-226. 

45. \citet{Steels:02c} Iterated Learning versus Language Games. Two models for cultural language evolution. In: Hemelrijk, C. and 
E. Bonabeau (ed.) Proceedings of the International Workshop of Self-Organization and Evolution of 
Social Behaviour, University of Zurich. 

46. \citet{Steels:02d} Bootstrapping grounded word semantics. In Briscoe, T. (ed.) Linguistic evolution through 
language acquisition: formal and computational models. Cambridge University Press. Cambridge, UK. 

47. \citet{Steels:02e} Crucial factors in the origins of word-meaning. In Wray, A (ed.)
The Transition to Language, Oxford University Press. Oxford. pp. 252-271.  

48. \citet{Steels:02f} Simulating the Evolution of a Grammar for Case. In: Proceedings of the Eurocores Conference 
on the Origins of Man, Language and Languages. MPI, Leipzig.

49. \citet{Steels:03a} Shared Grounding of Event Descriptions by Autonomous Robots. 
Robotics and Autonomous Systems, 43(2-3):163-173.

50. \citet{steels:03b} Evolving grounded communication for robots. Trends in Cognitive Science, 7(7):308-312. 

51. \citet{Steels:03c} Language-reentrance and the Inner Voice. Journal of Consciousness Studies, 10(4-5):173-185. 

52. \citet{Steels:03d} Intelligence with representation. Phil Trans Royal Soc Lond A., 361(1811):2381-2395.

53. \citet{Steels:03e} Steels, L. (2003) Social Language learning. In: Steels, L. and M. Tokoro (eds) The Future of Learning. 
IOS Press. Amsterdam. pp. 133-162. 

54. \citet{dejong:03f} A Distributed Learning Algorithm for Communication Development. 
Complex Systems, 14(4-5):315-334. 

55. \citet{Neubauer:03g}  Compositionality in Horizontal Transmission.  Poster European Cognitive Science Conference, 
University of Osnabrück. 

56. \citet{Manuel:03h} Creating a Robot Culture: An Interview with Luc Steels. IEEE Intelligent Systems, May/June 2003, pp. 59-61. 

57. \citet{steels:04a} The Evolution of Communication Systems by Adaptive Agents. In Alonso, E., D. Kudenko and D. Kazakov (eds.) Adaptive Agents and Multi-Agent Systems, Lecture Notes in AI. (vol. 2636), Springer-Verlag Berlin. pp. 125-140. 

58. \citet{Steels:04b} The Autotelic Principle. In Fumiya, I., R. Pfeifer, L. Steels, K. Kunyoshi (eds.) (2006) Embodied AI., Lecture Notes in AI (vol. 3139), Springer Verlag. Berlin. pp. 231-242. 

59. \citet{Steels:04c} The Architecture of Flow. In: Tokoro, M. and L. Steels, editor, A Learning Zone of One's Own, IOS Press. Amsterdam. 
pp. 135-150.

60. \citet{Steels:04d} Constructivist Development of Grounded Construction Grammars. In: D. Scott, W. Daelemans and M. Walker (Eds.), 
Proceedings Annual Meeting of Association for Computational Linguistics Conference. Barcelona: ACL. (pp. 9-16. 

61. \citet{Steels:04e} Fluid Construction Grammars. Poster Third International Conference in Construction Grammar. Marseille.  

62. \citet{Steels:05a} The Emergence and Evolution of Linguistic Structure: From Lexical to Grammatical Communication Systems. Connection Science. 17(3). 

63. \citet{Steels:2005} Coordinating Perceptually Grounded Categories through Language. A Case Study for Colour. Target article. Behavioral and Brain Sciences. 28 (4). 469-490. 

64. \citet{Steels:05c} The Semiotic Dynamics of Color. Behavioral and Brain Sciences 28(4). 

\enlargethispage{1\baselineskip}
65. \citet{Steels:05d} Linking in Fluid Construction Grammar. Proceedings of BNAIC. Transactions of the Belgian Royal Society for 
Science and Arts. pp. 11--18. October 2005.  

\enlargethispage{1em}
66. \citet{Steels:05e} Planning What To Say: Second Order Semantics for Fluid Construction Grammars. 
In: Bugarin Diz, A. and J. Santos Reyes (eds.) Proceedings of CAEPIA 2005. Springer-Verlag, Berlin. 

67. \citet{Steels:05f} (2005) What triggers the emergence of grammar?. In: Dautenhahn, K. (ed). (2005) AISB'05.  Proceedings of the 
Second International Symposium on the Emergence and Evolution of Linguistic Communication (EELC'05), University of Hertfordshire. Hatfield, pp. 143-150. 

68. \citet{Steels:05g} Hierarchy in Fluid Construction Grammar. 
In: Furbach, U. (eds) (2005) Proceedings of KI-2005. Lecture Notes in AI 3698. Springer-Verlag, Berlin. pp. 1-15. 

69. \citet{Steels:06a} Semiotic Dynamics for Embodied Agents. IEEE Intelligent Systems. May/june 2006, pp. 32-38. 

70. \citet{Steels:06b} Unify and Merge in Fluid Construction Grammar. In Vogt, P., Sugita, Y., Tuci, E. and Nehaniv, C., editor, Symbol Grounding and Beyond: Proceedings of the Third International Workshop on the Emergence and Evolution of Linguistic Communication, LNAI 4211, pages 197-223, Springer-Verlag. Berlin. 

71. \citet{Steels:06c} Collaborative Tagging as Distributed Cognition. Journal on 
Pragmatics and Cognition. 14:2 (2006), 287-292. 

72. \citet{Steels:06d} Experiments on the emergence of human communication. Trends in Cognitive Sciences, 10(8):347-349.

73. \citet{Steels:06e} How grammar emerges to dampen combinatorial search in parsing. In: Vogt, P. , Y. Suga, E. Tuci, C. Nehaniv (2006) (eds.) Symbol Grounding and beyond. LNAI 4211. Springer-Verlag, Berlin. pp. 76-88. 

74. \citet{Loreto:2007} Social Dynamics: Emergence of Language. Nature Physics 3(11) pp. 758-760. 

75. \citet{Steels:07b} Semiotic Dynamics solves the Symbol Grounding Problem. Nature Precedings 
\url{http://precedings.nature.com/documents/1234/version/1}

75. \citet{Steels:07c} Fifty Years of AI: From Symbols to Embodiment - and Back.  In: M. Lungarella et al. (Eds.): 50 Years of AI, Festschrift, LNAI 4850, Springer-Verlag Berlin, pp. 18-28. 

76. \citet{vantrijp:07d} Multi-Level Selection in the Emergence of Language Systematicity. In Almeida e Costa, F. and Rocha, L.M. and Costa, E. and Harvey, I (eds.) Proceedings of the Ninth European Conference on Artificial Life, LNCS 4648.
LNCS Springer Verlag, Berlin. pp. 425-434. 

77. \citet{Steels:07e} Steels, L. and P. Wellens (2007) Scaffolding Language Emergence Using the Autotelic Principle. 
First IEEE Alife Conference, Hawaii. IEEE Press, Wiley NY. pp. 325-332. 

78. \citet{Steels:07f} Language originated in social brains. In: O. Vilarroya and Francesc Forn (eds.) Stances on the Social Brain. VIBS. Rodopi, Barcelona. pp. 223-242. 

80. \citet{Steels:07g} The Recruitment Theory of Language Origins. In: Lyon, C., C. Nehaniv, and A. Cangelosi (eds) Emergence of Communication and Language. Springer Verlag, Berlin. pp. 129-151. 

81. \citet{Steels:07h} Is Symbolic Inheritance Similar to Genetic Inheritance?. Behavioral and Brain Sciences, 30(4):376-377.

\enlargethispage{2em}
82. \citet{Steels:07i} Embodiment and Self-Organization of Human Categories: A Case Study for Speech. In: Ziemke, T. and Zlatev, J. and R. Frank (eds.) Body, Language, and Mind, Cognitive Linguistics Research, Mouton De Gruyter. Berlin, pp. 411-430. 

83. \citet{Wellens:2008} Flexible Word Meaning in Embodied Agents. Connection Science, 20(2-3): 173-191. 

84. \citet{Wang:08b} Self-interested agents can bootstrap symbolic communication if they punish cheaters. In Smith, A. D. M. and K. Smith and R. Ferrer-i-Cancho, (eds.) The Evolution of Language. Proceedings of the 7th International Conference on the Evolution of Language, World Scientific, Singapore. pp. 362-369. 

85. \citet{Steels:2008} The Robot in the Mirror. Connection Science 20(2-3), 337-358. 

86. \citet{Steels:08d} Can Body Language Shape Body Image?. In Bullock, S. and Noble, J. and Watson, R. and 
Bedau, M. A., editor, Artificial Life XI: Proceedings of the Eleventh International Conference on the Simulation and 
Synthesis of Living Systems, The MIT Press, Cambridge Ma. pp. 577-584. 

87. \citet{Loetzsch:08e}  Typological and Computational Investigations of Spatial Perspective. In: Wachsmuth, I. 
and G. Knoblich (eds.), Modeling Communication with Robots and Virtual Humans, LNCS (vol. 4930). Springer-Verlag, Berlin. pp. 125-142. 

88. \citet{Steels:08f} Biological Roots of the Social Brain. Biological Theory, 3(1):93-98. 

89. \citet{Baronchelli:08g} In-Depth Analysis of the Naming Game Dynamics: The Homogeneous Mixing Case. International Journal of Modern Physics C, 19(5):785-812. 

90. \citet{Steels:08h}
Replicator Dynamics and Language Processing. In: Proceedings of the 7th Evolution of Language Conference, Barcelona. pp. 503. 

91. \citet{Galantucci:08i} The Emergence of Embodied Communication in Artificial Agents and Humans. In: Wachsmuth, I, et.al. (eds.) Embodied Communication. Oxford University Press, Oxford. 

92. \citet{Steels:08j} The Symbol Grounding Problem has been solved. So What's Next? In: Glenberg, A., A. Graesser, and M. de Vega (eds) (2008) Symbols, Embodiment and Meaning. Oxford University Press, Oxford. pp. 506-557. 

93. \citet{Steels:09a} Perspective Alignment in Spatial Language. In: Coventry, K., J. Bateman and T. Tenbrink (eds) (2009) Spatial Language in Dialogue. Oxford University Press. Oxford. 

94. \citet{Bleys:09b} The Grounded Color Naming Game. Proceedings of the 18th IEEE International Symposium on Robot and Human Interactive Communication (Ro-man 2009). 

95. \citet{Goldstone:09c} The Emergence of Collective Structures Through Individual Interactions. In Taatgen, N.A. and van Rijn, H., editor, Proceedings of the 31th Annual Conference of the Cognitive Science Society, 2009.

96. \citet{Steels:09d}
Steels, L. (2009) Adaptive Language Games with Robots. American Institute of Physics. Proc. 1303, pp. 3-14

97. \citet{Gerasymova:09e} (2009) Aspectual Morphology of Russian Verbs in Fluid Construction Grammar. In Taatgen, N.A. and van Rijn, H., editor, Proceedings of the 31th Annual Conference of the Cognitive Science Society, pp. 1370-1375. 

98. \citet{Steels:09f} (2009) The relevance of agent-based language evolution models for archeology. In: J-M Hombert and F. d'Errico (eds) (2009) Becoming Eloquent. John Benjamins, Amsterdam. pages 267-286

99. \citet{Steels:09g} Is sociality a crucial prerequisite for the origins of language? In: Botha, R. and C. Knight (2009)
(ed.) The prehistory of language. Oxford University Press, Oxford. 

% 9h is identical to 10e

101. \citet{Steels:09i} How experience of the body shapes language about space. In Proceedings of 
IJCAI-09. Pasadena, Ca. AAAI Press, pp. 14-19.  

102. \citet{Steels:09j} Can Agent-Based Language Evolution Contribute to Archeology?. In d'Errico, F. and Hombert, J.M., editor, Becoming Eloquent: Advances in the Emergence of Language, Human Cognition, and Modern Cultures. John Benjamins. Amsterdam. pp. 267-286

103. \citet{Steels:09k} Cognition and social dynamics play a major role in the formation of grammar. In: D. Bickerton and E. Szathmary (2008) (eds.) Biological Foundations and Origin of Syntax. Strungmann Forum Reports, vol. 3. The MIT Press, Cambridge Ma. 

104. \citet{Steels:10a} Modeling the Formation of Language in Embodied Agents: Methods, Open Challenges, and Embodied Experiments. In: Nolfi, S. and M. Mirolli (eds) (2010) Evolution of Communication and Language in Embodied Agents. Springer Verlag, Berlin. p. 345-368. 

105. \citet{Steels:10b} Modeling the Formation of Language in Embodied Agents: Methods and Open Challenges. In Nolfi, S. and Mirolli, M., editor, Evolution of Communication and Language in Embodied Agents. Springer-Verlag, Berlin. pp. 223-233. 

106. \citet{Steels:10c} Modeling the Formation of Language: Embodied Experiments. In Nolfi, S. and Mirolli, M (eds.) Evolution of Communication and Language in Embodied Agents, Springer-Verlag. Berlin. pp. 235-262. 

107. \citet{Steels:10d} Modeling the Formation of Language: Conclusions and Future Research. In: Nolfi, S. and Mirolli, M., editor, Evolution of Communication and Language in Embodied Agents, Springer-Verlag. Berlin. pp. 283-288. 

108. \citet{Steels:10e} Babel: A Tool for Running Experiments on the Evolution of Language. In Nolfi, S. and Mirolli, M., editor, Evolution of Communication and Language in Embodied Agents, Springer-Verlag. Berlin. pp. 307-313.

109. \citet{sole:10f} Language Networks: Their Structure, Function, and Evolution. Complexity, 15(6):20-26. 

110. \citet{Bleys/Steels:11a} Linguistic Selection of Language Strategies. Advances in Artificial Life. Darwin Meets von Neumann Lecture Notes in Computer Science Volume 5778, 2011, pp 150-157

111. \citet{Steels:11b} Why We Need Evolutionary Semantics. KI 2011: Advances in Artificial Intelligence.  LNCS 7006. pp. 14-25. 

112. \citet{Steels:11c} Modeling the cultural evolution of language. Physics of Life Reviews. 8(4) 330-356. \url{http://www.sciencedirect.com/science/article/pii/S157106451100145X}

\newpage
113. \citet{Steels:11d} Design Patterns in Fluid Construction Grammar. John Benjamins Pub. Amsterdam 

114. \citet{Steels:11e} Introducing Fluid Construction Grammar. In: Steels, L. (ed.) (2011) Design Patterns in Fluid Construction Grammar. John Benjamins Pub. Amsterdam. pp. 3-30

115. \citet{Steels:11f} A first encounter with Fluid Construction Grammar. In: Steels, L. (ed.) (2011) Design Patterns in Fluid Construction Grammar. John Benjamins Pub. Amsterdam. pp. 31-68. 

116. \citet{Steels:11g} A design pattern for phrasal constructions. In: Steels, L. (ed.) (2011) Design Patterns in Fluid Construction Grammar. John Benjamins Pub. Amsterdam. pp. 69-114. 

117. \citet{Steels:11h} How to make construction grammars fluid and robust. In: Steels, L. (ed.) (2011) Design Patterns in Fluid Construction Grammar. John Benjamins Pub. Amsterdam. pp. 301-330. 

118. \citet{Vantrijp:12a} Multilevel alignment maintains language systematicity. Advances in Complex Systems. 15(3/4). 

119. \citet{Steels:12b} Evolutionary Language Games as a paradigm for Integrated AI research. In: 2012 AAAI Spring Symposium, AAAI Press. 
Anaheim Ca. 

120. \citet{Steels:12c} Computational Issues in Fluid Construction Grammar. LNAI 7249. Springer-Verlag, Berlin. 

121. \citet{Steels:12d} Language Grounding in Robots. Springer-Verlag, New York. 

122. \citet{Steels:12e} Experiments in Cultural Language Evolution. John Benjamins Pub., Amsterdam. 
 
123. \citet{Steels:12f} Grounding language through evolutionary language games. In: Steels, L. and M. Hild (eds.) (2012) Language Grounding in Robots. Springer-Verlag, New York. pp. 1-23.

124. \citet{Spranger:2012vision} A perceptual system for language game experiments. In: In: Steels, L. and M. Hild (eds.) (2012) Language Grounding in Robots. Springer-Verlag, New York. pp. 95-116. 

125. \citet{Spranger:2012} Open-ended Procedural Semantics. In: Steels, L. and M. Hild (eds.) (2012) Language Grounding in Robots. Springer-Verlag, New York. pp. 159-178. 

126. \citet{Steels:12i} Fluid Construction Grammar on Real Robots. In: Steels, L. and M. Hild (eds.) (2012) Language Grounding in Robots. Springer-Verlag, New York. pp. 201-219. 

127. \citet{Steels:2012b} Emergent action language on real robots. In: Steels, L. and M. Hild (eds.) (2012) Language Grounding in Robots. Springer-Verlag, New York. pp. 261-282. 

128. \citet{Steels:12k} Self-organization and selection in language evolution. In: Steels, L. (ed) Experiments in Cultural Language Evolution. John Benjamins. pp. 1-37. 

129. \citet{Steels:12l} The Grounded Naming Game. In: Steels, L. (ed) Experiments in Cultural Language Evolution. John Benjamins Pub., Amsterdam. pp. 41-59. 

\newpage
130. \citet{Steels:12m} Emergent mirror systems for body language. In: Steels, L. (ed) Experiments in Cultural Language Evolution. John Benjamins Pub., Amsterdam. pp. 87-109. 

131. \citet{spranger:12n} Emergent functional grammar for space. In: Steels, L. (ed) Experiments in Cultural Language Evolution. John Benjamins Pub., Amsterdam. pp. 207-232. 

132. \citet{beuls:12o} The emergence of internal agreement systems. In: Steels, L. (ed) Experiments in Cultural Language Evolution. John Benjamins Pub., Amsterdam. pp. 233-256. 

133. \citet{Steels:12p}  Design methods for Fluid Construction Grammar. In: Steels, L. ed. (2012) Computational Issues in Fluid Construction Grammar. Springer-Verlag, Berlin.  

134. \citet{Vantrijp:12q} Fluid Construction Grammar: The New Kid on the Block. Proceedings of the 13th Conference of the European Chapter of the Association for Computational Linguistics, Avignon, 2012 ACL.

135. \citet{Beuls:2013} Agent-Based Models of Strategies for the Emergence and Evolution of Grammatical Agreement. PLOS ONE, 8(3), e58960. \url{http://dx.plos.org/10.1371/journal.pone.0058960}

136. \citet{Steels:13b} (2013) How language emerges in situated embodied interactions. In: Lefebvre, C., B. Comrie (2013) New Perspectives on the Origins of Language. John Benjamins Publishing, Amsterdam. 

137. \citet{Steels:13c} Fluid Construction Grammar. In Hoffmann, T. and G. Trousdale (ed.) (2013) Handbook of Construction Grammar. Oxford University Press, Oxford. 

138. \citet{wellens:13d} Fluid Construction Grammar for Historical and Evolutionary Linguistics. Proceedings of the 51st Annual Meeting of the Association for Computational Linguistics, pages 127-132, Sofia, 2013 Association for Computational Linguistics.

139. \citet{Steels:14a} Breaking down barriers. In: Dor, D., C. Knight and J. Lewis (2014) The Social Origins of Language: Early Society, Communication and Polymodality. Oxford University Press, Oxford. 

140. \citet{Garcia:14b} Strategies for the emergence of first-order constituent structure. Proceedings of 
Evolang X, Vienna. 

141. \citet{Steels:14c} Crossing complexity barriers with grammar. A case study for agreement systems. In: Mufwene, S., J-M. Hombert, F. Pellegrino and C. Coupe (eds) Complexity in Language: Developmental and Evolutionary Perspectives. Cambridge University Press, Cambridge. 
  