\addcontentsline{toc}{chapter}{Preface}
\chapter*{Preface}

I set up the Talking Heads experiment with a group of brilliant students and collaborators at the end of the nineteen nineties. It was intended to be the first large-scale, open-ended experiment in the emergence of a shared set of grounded concepts and a vocabulary for expressing these concepts by a population of autonomous agents. Inspired by Ludwig Wittgenstein, the experiment took the form of a series of language games, more concretely games of reference about a ``world'' made up of geometric figures pasted on a white board and observable by the agents through pan-tilt cameras. 
I  wanted to demonstrate with this experiment earlier breakthroughs in the study of language origins and 
test whether they would hold for large-scale populations and open-ended environments. I also wanted to find 
out how humans would interact with these agents. So we made it such that human users, after logging in through the 
Internet, could teach new words to agents or use the words they learned from the agents to play their own language games. 

In 1999, the experiment went live in the context of an exhibition called Laboratorium organised in 
Antwerp (Belgium) by Hans-Ulrich Obrist and Barbara Vanderlinden. After a first experimental run from 27 June 1999 to 
3 October 1999, the experiment was repeated as part of a new exhibition called N01SE organised in Cambridge and 
London (UK) by Adam Lowe and Simon Schaffer from 22 January to 26 March 2000, with additional installations at 
the Palais de la D\'{e}couverte in Paris and several other places. 

At the occasion of the 1999 Laboratorium exhibition, the draft of a book was published 
that described the experiment and the underlying theoretical assumptions in considerable detail. 
For many reasons, not at least 
that work continued at great speed on other exciting experiments, the original ``pre-edition'' of the
book never made it to a fully finished officially 
published work, and circulated only as an ``underground'' edition. 
This was disappointing because the Talking Heads experiment was an important breakthrough. 
Moreover the experiment contained the first inklings of mechanisms that since 
then have been worked out, enhanced and tested in many experiments which replicated the original results and further
enhanced them. The present volume is intended to fill this gap. 

\partref{part:1} of this book contains the original Talking Heads volume which has been only 
slightly edited to correct for minor mistakes. It is a miracle that the original source files survived and that the 
figures could be reconstructed. \partref{part:2} of this book contains additional unpublished background 
material, reports on different aspects of the experiment, including its scientific results, and 
a brief overview of further developments in language evolution research that took place on the basis of the experiment. 

Dozens of people worked on the Talking Heads. The initial research started in 1997 at the Artificial Intelligence 
Laboratory of the Free University of Brussels (VUB) funded by a ``Geconcentreerde onderzoeks actie'' (GOA)
of the Belgian government. Joris Van Looveren worked closely with me on a first prototype 
that was demonstrated in 1997, using an active vision system custom-built by Tony Belpaeme and segmentation 
algorithms implemented by Danny van Tieghem. Edwin de Jong did the first theoretical investigations of the underlying 
semiotic dynamics. Once the initiative for a public installation was taken,  
the main hub became the Sony Computer Science Laboratory in Paris, 
where I worked together intensely with Fr\'{e}d\'{e}ric Kaplan and Angus McIntyre, with additional contributions for the 
teleportation infrastructure by Silv\`{e}re Tajan and Alexis Agahi. The AI Laboratory of the VUB remained
a second hub where important contributions were 
made most notably by Joris Van Looveren, Tony Belpaeme, Holger Kenn and Mario Campanella. To all of them I 
am grateful that we were able to create such an extremely exciting experiment that stimulated many 
thousands of people to think about language and its origins in new ways. 

I also thank the curators of the Laboratorium Exhibition in Antwerp (Hans-Ulrich Obrist and Barbara Vanderlinden) 
and the members of its curating board (Bruno Latour and Carsten H\"oller), the curators of the N01SE exhibition 
in London (Adam Lowe and Simon Schaffer), artist Olafur Eliasson, with whom I collaborated on the Look into the Box 
piece at the Mus\'e d'art Moderne in Paris, artist Anne-Mie Van Kerckhoven for collaborations for the Chromosophy 
laboratory in Aachen and the many organisers and helping hands who made the other installations possible. 
Sylvia Spruck Wrigley has helped to improve the 1999 pre-edition (now \partref{part:1} of the book) 
under considerable time pressure and Marleen Wynants was crucial in the last phases of this publication with comments,
professional advice, photographs, and support. 

\partref{part:2} of the book discusses what happened after the Talking Heads experiment. Our research went through various 
boom and bust cycles. In the good years there was money to hire new people and push the research forwards, but then 
bad years would come and the team disintegrated again due to lack of resources. This made progress less substantial than it
could have been but had the advantages that waves of new young people were given a chance to contribute. The first wave 
started working after the Talking Heads experiments were finished. At Sony CSL there was first exciting work by Pierre-Yves Oudeyer 
and Frederic Kaplan pursuing the origins of turn taking and symbol usage with the \textsc{aibo} robots, thus exploring even earlier 
stages in the origins of language. 

Thanks to the FP6 ECAgents project and the FP7 ALEAR projects of the European Commission, a new 
team could be formed around 2004, which included Joris Bleys, Joachim De Beule, Bart de Vylder, and Jelle Zuidema at the 
VUB in Brussels. Meanwhile, we got access through the Sony Computer Science Laboratory to the QRIO humanoid robots 
thanks to Masahiro Fujita and Hideki Shimomura. A new team formed in Paris which included Nancy Chang, Katya Gerasimova, Martin Loetzsch, 
Vanessa Micelli, Michael Spranger, Simon Pauw and Remi Van Trijp. I thank them all for major contributions to the experiments reported 
in the second part of the book. I also thank Stefano Nolfi for coordinating the ECAgents project with a firm hand 
and Manfred Hild and his team for creating the new humanoid \textsc{myon} robot that we used in later experiments. 

In 2008/2009 I was a fellow at the Wissenschaftskolleg in Berlin, which was an ideal environment to reflect on our research 
program and how we could proceed. Many new ideas came out of that year, in particular, a realisation that for several of the key
puzzles we were trying to solve, inspiration could be found in evolutionary biology, and this led to a new path which only now 
is beginning to be explored. It is good to know that after a major collapse in funding around 2009, there is 
again a team of young people assembling to push research on language games further. They include Emilia Garcia Casademont in 
Barcelona, Miquel Cornudella and Paul Van Eecke in Paris, and Yana Knight in Brussels. Financing remains precarious 
but the future is in their hands! 

The present new edition is not only special because of the historical significance of the Talking Heads 
experiment but also because it is the first one in a new series ``Computational Models of 
Language Evolution'' published by the Language Science Press. The intention of this series is to make available 
through Open Access in-depth models of language evolution that have been validated using agent-based 
computational simulations. I am grateful to Martin Haspelmath and Stefan M\"uller (the editors of the Language Science Press)
for making it possible to publish in Open Access research results which do not quite fit in the standard mode. I thank 
Maria Ferrer Bonnet for help in the final stage of adapting bibliographies. And I am grateful to ICREA for time to 
create a revised and extended version of the Talking Heads Book and to the Institut de Biologia Evolutiva in Barcelona 
for providing such an excellent working environment. I thank in addition Remi van Trijp for his help in making this series a reality 
and Annemie Maes for her encouragement in the final stretches to finish this book. 

\vspace{1cm}

\hfill Barcelona, October 2014.  


