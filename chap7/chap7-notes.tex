
\bigskip
\paragraph{\bf Notes}
\paragraph{}
\footnotesize
\vspace{0.2cm}
\newline
\hspace{-0.5cm}
?. Wittgenstein
\vspace{0.2cm}
\newline
\hspace{-0.5cm}
?. spin glass models. 
\vspace{0.2cm}
\newline
\hspace{-0.5cm}
?. The following channels are available to the agents 
in the experiments in this section: 
HPOS (the horizontal position of the midpoint 
of the segment's bounding box), 
VPOS (the vertical position of the midpoint of the segment's 
bounding box), HEIGHT (the height of the bounding box), 
WIDTH (the width of the bounding box), AREA
(the area of the segment, calculated
by counting the number of pixels that belong to it), 
R (the average redness of the pixels in the segment), 
G (the average greenness of the pixels in the segment), 
B (the average blueness of the pixels in the segment). 
These `color' channels are not to be confused with the 
human opponent color channels so the distinctions 
that form of them are not directly perceivable by 
human observers. 
\vspace{0.2cm}
\newline
\hspace{-0.5cm}
?. The different sensory values (after sensor-scaling)
for the segments in game 3 are shown in the table below. 
\begin{center}
\begin{tabular}{ l  l  l } \lsptoprule
{\itshape channel}& {\itshape obj-0} & {\itshape obj-1}\\ \midrule
HPOS & 0.27 & 0.16\\ 
VPOS & 0.20 & 0.20\\ 
HEIGHT & 0.15 & 0.15\\ 
WIDTH & 0.10 & 0.11\\ 
AREA & 0.10 & 0.10\\ 
R & 0.23 & 0.25\\ 
G & 0.32 & 0.34\\ 
B & 0.63 & 0.65\\ 
\lspbottomrule
\end{tabular}
\end{center}
HPOS is the most salient channel. After sensor-scaling, 
the two values for HPOS are still be drawn further
apart with 1.0 for object-0 and 
0.0 for object-1 so that the category [HPOS 0.5-1.0] easily 
distinguishes the topic (object-0) from object-1. 
\vspace{0.2cm}
\newline
\hspace{-0.5cm}
?. The segments and sensory data for game 126 
(after sensor-scaling) are shown in the following table: 
\begin{center}
\begin{tabular}{ l  l  l } \lsptoprule
{\itshape channel}& {\itshape obj-0} & {\itshape obj-1}\\ \midrule
HPOS & 0.08 & 0.09\\ 
VPOS & 0.21 & 0.10\\ 
HEIGHT & 0.10 & 0.15\\ 
WIDTH & 0.42 & 0.11\\ 
AREA & 0.34 & 0.16\\ 
R & 0.33 & 0.31\\ 
G & 0.68 & 0.65\\ 
B & 0.52 & 0.50\\ 
\lspbottomrule
\end{tabular}
\end{center}
Clearly width is the most salient channel for the speaker. 
\vspace{0.2cm}
\newline
\hspace{-0.5cm}
?. The segments and sensory data of the speaker
in game 128 (after sensor-scaling) are shown in the 
following table: 
\begin{center}
\begin{tabular}{ l  l  l } 
\lsptoprule
{\itshape channel}& {\itshape obj-0} & {\itshape obj-1} & {\itshape obj-2}\\ \midrule
HPOS & 0.03 & 0.15 & 0.33\\ 
VPOS & 0.29 & 0.29 & 0.01\\ 
HEIGHT & 0.37 & 0.39 & 0.07\\ 
WIDTH & 0.10 & 0.08 & 0.27\\ 
AREA & 0.29 & 0.22 & 0.16\\ 
R & 0.95 & 0.97 & 0.97 \\ 
G & 0.33 & 0.40 & 0.93\\ 
B & 0.35 & 0.44 & 0.26\\ 
\lspbottomrule
\end{tabular}
\end{center}
\vspace{0.2cm}
\newline
\hspace{-0.5cm}
?. tools? 
\vspace{0.2cm}
\newline
\hspace{-0.5cm}
?. The website http://talking-heads.csl.sony.fr/ 
contains the latest statistics on this world-wide evolution. 
The tools were mainly built by Joris Van Looveren and 
Frederic Kaplan. 
All data
are taken from experiments with the Talking Heads as
they play grounded language games. 

To enable a systematic study of the unfolding semiotic dynamics, 
we have built tools that track the language games 
as they take place in parallel on a world-wide scale. The tools 
are available to anyone who logs on through the Internet
and wants to see for him or herself how the language
system is evolving.$^?$ The tools record the 
situations, images, words, and meanings that have 
effectively been used in particular games at each 
physical site. A situation corresponds to the 
choice for a particular topic in a particular
context. When the situations are 
sufficiently similar in physical terms 
they are considered to be identical. The meaning and the 
image constructed by each agent are also 
recorded, even though a linguist would never be able to 
do so. For the rest of this chapter, I will ignore 
the complexity due to perceptual anomalies between 
the speaker and the hearer's segmentation and sensory 
processing of the images. 

\vspace{0.2cm}
\newline
\hspace{-0.5cm}
?. 
\vspace{0.2cm}
\newline
\hspace{-0.5cm}
?. debate on functionality of language. Ideal language. 
cf. chomsky, kirby 
\vspace{0.2cm}
\newline
\hspace{-0.5cm}
?. eco = search for ideal language
\vspace{0.2cm}
\newline
\hspace{-0.5cm}
?. This work was done in strong collaboration with 
Frederic Kaplan and first published as: 
Steels, L. and F. Kaplan (1999) Collective Learning and Semiotic 
Dynamics. In: {\itshape Proceedings of the European Conference on 
Artificial Life 99, Lausanne}. The MIT Press, Cambridge. 

\bigskip
\paragraph{\bf Selected Readings}
\paragraph{}
\footnotesize

\hspace{-0.5cm}
Clark, E. (1979) The Ontogenesis of Meaning. Akademische
Verlagsgesellschaft Athenaion, Wiesbaden. 
\vspace{0.2cm}
\newline
\hspace{-0.5cm}
Quine, W.V.O. (1960) Word and Object. The MIT Press, 
Cambridge Ma.
\normalsize
\end{document}